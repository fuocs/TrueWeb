This chapter describes the trustworthiness scoring mechanism used by TrueWeb to evaluate website safety. The scoring system aggregates multiple security indicators into a single numerical score that reflects the overall reliability of a website.

\section{Scoring Design Overview}

Designing an effective website safety scoring system requires balancing two competing goals: accuracy (capturing the true risk level) and interpretability (presenting results that users can understand and act upon). TrueWeb addresses this challenge through a weighted multi-criteria model that combines diverse security signals while remaining transparent about how the final score is derived.

\subsection{Design Rationale}

The scoring system is built on several key principles:

\begin{itemize}
    \item \textbf{Multi-dimensional Assessment:} No single indicator can reliably capture all aspects of website safety. By evaluating nine distinct criteria, the system reduces the risk of false positives or negatives that might occur when relying on a single data source.
    
    \item \textbf{Weighted Importance:} Different security factors carry different levels of significance. For example, a website flagged by multiple reputation databases represents a more serious concern than minor HTML anomalies. The weighting scheme reflects these relative priorities.
    
    \item \textbf{Graceful Degradation:} When certain data sources are unavailable (e.g., API rate limits, network errors), the system continues to function using the remaining modules rather than failing entirely.
    
    \item \textbf{Human-Interpretable Scale:} The final score uses a familiar 0.0--5.0 scale (analogous to star ratings), making it immediately understandable to non-technical users.
\end{itemize}

\subsection{Score Range and Interpretation}

The final trust score ranges from 0.0 (highly dangerous) to 5.0 (highly reliable). This scale was chosen to provide sufficient granularity for distinguishing between risk levels while remaining intuitive. A score of 5.0 indicates that all evaluated criteria returned positive results, while a score approaching 0.0 suggests multiple serious red flags across different analysis modules.

\section{Scoring Criteria and Weights}

Each security criterion contributes differently to the final trust score according to its relative importance. The weighting scheme is designed to balance four main categories of indicators:

\begin{enumerate}
    \item \textbf{Historical Reputation} (highest weight): External databases that track known threats provide the most reliable signals, as they aggregate reports from security researchers and organizations worldwide.
    
    \item \textbf{AI-based Analysis} (high weight): Machine learning models can detect sophisticated patterns that rule-based systems might miss, particularly for novel phishing attempts or content-based threats.
    
    \item \textbf{Technical Indicators} (moderate weight): Protocol security, certificate validity, server behavior, and domain characteristics provide objective technical signals about website trustworthiness.
    
    \item \textbf{Supplementary Signals} (lower weight): Community reviews offer valuable real-world feedback but are assigned lower weight due to potential manipulation or limited coverage for lesser-known websites.
\end{enumerate}

The system evaluates websites using nine distinct scoring modules. The weight assigned to each module reflects both its reliability and its discriminative power in distinguishing safe websites from malicious ones:

\begin{table}[H]
\centering
\begin{tabular}{|l|c|l|}
\hline
\textbf{Criterion} & \textbf{Weight} & \textbf{Description} \\
\hline
Reputation Databases & 2.0 & VirusTotal, Google Safe Browsing lookup \\
AI Analysis & 1.5 & Groq AI content evaluation \\
Domain Age & 1.0 & WHOIS registration age analysis \\
Server Reliability & 0.8 & IP stability, hosting, redirections \\
Domain Pattern & 0.8 & Typosquatting, suspicious TLDs \\
Protocol Security & 0.8 & HTTPS enforcement, secure headers \\
HTML Content \& Behavior & 0.7 & Form actions, obfuscation, hidden elements \\
Certificate Details & 0.6 & SSL/TLS validity, issuer verification \\
User Reviews & 0.1 & Community-driven safety ratings \\
\hline
\end{tabular}
\caption{Scoring criteria and their corresponding weights}
\label{tab:scoring-weights}
\end{table}

\begin{figure}[H]
    \centering
    \includegraphics[width=0.5\linewidth]{Weight.png}
    \caption{Trustworthiness scoring factors and corresponding weights}
    \label{fig:trust-score-weights}
\end{figure}

\section{Score Aggregation and Normalization}

Each evaluation criterion produces a subscore that reflects how well the website satisfies a specific safety requirement. The final trust score is computed by aggregating all subscores using the following formula:

\[
\text{Trust Score} =
\frac{\sum (\text{weight} \times \text{subscore})}{\sum (\text{weights})} \times 5
\]

This normalization ensures that the final score is consistently mapped to a 0.0--5.0 scale regardless of the number of criteria used.

\section{Risk Level Classification}

To translate the numerical trust score into actionable guidance, TrueWeb categorizes websites into three risk levels based on predefined thresholds:

\begin{table}[H]
\centering
\begin{tabular}{|c|l|p{8cm}|}
\hline
\textbf{Score Range} & \textbf{Classification} & \textbf{Recommended Action} \\
\hline
4.0 -- 5.0 & \textcolor{green}{Trusted} & The website shows no significant risk indicators. Users may proceed with normal caution. \\
\hline
3.0 -- 4.0 & \textcolor{orange}{Caution} & Some risk factors were identified. Users should review the detailed breakdown before proceeding and avoid entering sensitive information. \\
\hline
0.0 -- 3.0 & \textcolor{red}{Potentially Unsafe} & Multiple risk indicators detected. Users are strongly advised to avoid this website or proceed only if absolutely necessary with heightened vigilance. \\
\hline
\end{tabular}
\caption{Risk level classification and recommended user actions}
\label{tab:risk-levels}
\end{table}

These thresholds were calibrated based on empirical testing with known safe and malicious websites. The relatively high threshold for ``Trusted'' status (4.0) ensures that only websites passing most evaluation criteria receive a clean assessment, while the ``Caution'' range (3.0--4.0) captures borderline cases that warrant user attention without triggering unnecessary alarm.

\section{Handling Missing or Failed Modules}

In practice, certain scoring modules may occasionally fail due to external factors such as API rate limits, network timeouts, or unavailable data sources. To maintain robustness, TrueWeb implements the following fallback behavior:

\begin{itemize}
    \item \textbf{Retry Mechanism:} Failed modules automatically retry with exponential backoff (configurable retry count) before being marked as failed.
    
    \item \textbf{Neutral Default:} If a module fails after all retry attempts, it contributes a neutral subscore (0.5) rather than being excluded entirely. This prevents artificially inflated or deflated scores due to missing data.
    
    \item \textbf{Transparency:} Failed modules are clearly indicated in the detailed breakdown, allowing users to understand which data sources were unavailable during the analysis.
\end{itemize}

\section{Chapter Summary}

The trustworthiness scoring system provides a structured and transparent framework for evaluating website safety. By combining nine distinct analysis modules with carefully calibrated weights, TrueWeb produces a comprehensive safety assessment that balances accuracy with interpretability. The weighted aggregation approach ensures that critical risk indicators (such as reputation database flags) have greater influence on the final score, while supplementary signals (such as user reviews) provide additional context without dominating the assessment.

The 0.0--5.0 scoring scale and three-tier risk classification system enable users to quickly understand the safety status of a website, while the detailed breakdown provides transparency for users who wish to examine specific risk factors. This design supports informed decision-making across a wide range of user technical proficiencies.
