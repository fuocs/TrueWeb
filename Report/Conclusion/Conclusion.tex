\section{Conclusion}
\label{sec:conclusion}

This project introduced \textbf{TrueWeb}, a multi-layer website safety assessment system that assists users in evaluating the trustworthiness of a URL before interacting with potentially harmful web content. The core idea of TrueWeb is to integrate multiple independent sources of evidence---including \textit{protocol and certificate signals, domain characteristics, server-related information, HTML behavioral heuristics, reputation databases, AI-based content analysis, and crowdsourced reviews}---into a unified trust score that is both interpretable and actionable.

From an engineering perspective, the system was implemented as a modular desktop application using PyQt6, supported by a Python-based analysis engine. To enhance usability and reduce user friction, TrueWeb incorporates a browser extension that forwards URLs directly to the desktop application through a local-only communication channel. The backend design emphasizes practical robustness through shared HTML fetching, parallel module execution, and retry mechanisms with exponential backoff to handle transient API and network failures.

The evaluation results, conducted on a balanced dataset of phishing and legitimate URLs, demonstrate that combining heterogeneous signals provides more reliable phishing detection than relying on any single feature category. In particular, the ablation study indicates that high-impact criteria such as reputation databases and content or behavior-based signals contribute significantly to discriminating phishing websites, while lower-impact criteria can be assigned smaller weights to reduce false positives.

In addition to detection performance, TrueWeb places strong emphasis on user experience. The system provides real-time URL analysis, visualized safety scoring, and clear progress indicators, supported by a minimalist and intuitive interface with dark/light mode support and flexible authentication options. User feedback suggests that this design improves accessibility and encourages trust in the system’s recommendations.

Overall, TrueWeb demonstrates that a practical, user-facing phishing risk assessment tool can be built using a principled multi-criteria scoring framework, supported by an extensible module architecture, responsive user interface, and community-aware design. The system is particularly suitable as a browsing safety assistant in environments where users frequently encounter unfamiliar or suspicious links.
