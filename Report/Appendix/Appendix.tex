\chapter{Appendix}

\section{Project Source Structure}
This appendix summarizes the main source folders of the \textbf{TrueWeb} project and their responsibilities.

\begin{itemize}
    \item \textbf{backend/}: Core website analysis and scoring logic.
    \begin{itemize}
        \item Reputation checks (e.g., Google Safe Browsing / VirusTotal integration).
        \item URL parsing, pattern analysis, WHOIS lookup, SSL/TLS checks.
        \item Screenshot capture and HTML heuristic analysis.
        \item Trust score aggregation.
    \end{itemize}

    \item \textbf{frontend/}: Desktop UI (PyQt6) and user interaction.
    \begin{itemize}
        \item Main window, loading/result pages, settings.
        \item System tray behavior and notification flow.
        \item User review and feedback UI.
    \end{itemize}

    \item \textbf{extension/}: Browser extension for quick URL collection.
    \begin{itemize}
        \item Captures/forwards URLs to the local desktop integration server.
        \item Supports global shortcut integration (depending on browser permissions).
    \end{itemize}
\end{itemize}

\section{Key Runtime Behavior}
\begin{itemize}
    \item \textbf{Input:} URL collected from manual input or browser extension.
    \item \textbf{Processing:} Backend modules fetch metadata (WHOIS/SSL/IP/HTML/reputation) and compute a trust score on a 0.0--5.0 scale.
    \item \textbf{Output:} UI displays score, risk indicators, and recommended actions.
\end{itemize}

\section{Keyboard Shortcut (Extension)}
Depending on the browser and extension configuration, TrueWeb supports a shortcut such as \texttt{Ctrl+Shift+Space} for quick URL forwarding.

\section{Build and Run Notes}
\begin{itemize}
    \item \textbf{Dependencies:} Python 3.x and required packages listed in \texttt{requirements.txt}.
    \item \textbf{Run (development):} Start the application entry point (e.g., \texttt{main.py}) after installing dependencies.
    \item \textbf{Packaging:} The project can be bundled into an executable using \textbf{PyInstaller} (see the provided spec file if available).
    \item \textbf{Configuration:} API keys and environment variables should be managed via a \texttt{.env} file based on \texttt{backend/.env.example}.
\end{itemize}
